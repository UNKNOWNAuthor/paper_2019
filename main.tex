\documentclass[sigconf,edbt]{acmart-edbt2019}

\usepackage{booktabs} % For formal tables
\usepackage{subfigure}
\usepackage{color}
\usepackage{amsmath}
\usepackage{graphicx}
\usepackage{caption}

\usepackage{caption}
\usepackage{color}
\usepackage[ruled]{algorithm}
\usepackage{algorithmic}

\newcommand{\ie}[0]{\textit{i.e.}}
\newcommand{\eg}[0]{\textit{e.g.}}
\newcommand{\etal}[0]{\textit{et al.}}
\newcommand{\wrt}[0]{\textit{w.r.t.}}


\newcommand{\keobf}{\textit{$(k,\epsilon)$-obf}}
\newcommand{\argmin}{\operatornamewithlimits{argmin}}
\newcommand{\genobf}{\texttt{\textbf{genObf}}}
\newcommand{\ere}{$\mathcal{E}RR-$eval}
\newcommand{\pg}{\mathcal{G}}
\newcommand{\Constraint}{$\mathcal{C}$}
\newcommand{\Number}{\|v\|}

\theoremstyle{plain}
\newtheorem{problem}{Problem}
\newtheorem{observation}{Observation}

\newcommand{\methodName}{Squid}
\newcommand{\capMethodName}{SQUID}
\newcommand{\soaName}{Obf}


% Copyright
\setcopyright{rightsretained}

% DOI
\acmDOI{}

% ISBN
\acmISBN{XXX-X-XXXXX-XXX-X}

%Conference
\acmConference[EDBT 2019]{22nd International Conference on Extending Database Technology (EDBT)}{March 26-29, 2019}{Lisbon, Portugal} 
\acmYear{2019}

\settopmatter{printacmref=false, printccs=false, printfolios=false}

\pagestyle{empty} % removes running headers


\begin{document}
\title{SIG Proceedings Paper in LaTeX Format}
\titlenote{Produces the permission block, and copyright information}
\subtitle{Extended Abstract}
\subtitlenote{The full version of the author's guide is available as
  \texttt{acmart.pdf} document}
  

\author{Ben Trovato}
\authornote{Dr.~Trovato insisted his name be first.}
\orcid{1234-5678-9012}
\affiliation{%
  \institution{Institute for Clarity in Documentation}
  \streetaddress{P.O. Box 1212}
  \city{Dublin} 
  \state{Ohio} 
  \postcode{43017-6221}
}
\email{trovato@corporation.com}

\author{G.K.M. Tobin}
\authornote{The secretary disavows any knowledge of this author's actions.}
\affiliation{%
  \institution{Institute for Clarity in Documentation}
  \streetaddress{P.O. Box 1212}
  \city{Dublin} 
  \state{Ohio} 
  \postcode{43017-6221}
}
\email{webmaster@marysville-ohio.com}

\author{Lars Th{\o}rv{\"a}ld}
\authornote{This author is the
  one who did all the really hard work.}
\affiliation{%
  \institution{The Th{\o}rv{\"a}ld Group}
  \streetaddress{1 Th{\o}rv{\"a}ld Circle}
  \city{Hekla} 
  \country{Iceland}}
\email{larst@affiliation.org}

\author{Valerie B\'eranger}
\affiliation{%
  \institution{Inria Paris-Rocquencourt}
  \city{Rocquencourt}
  \country{France}
}
\author{Aparna Patel} 
\affiliation{%
 \institution{Rajiv Gandhi University}
 \streetaddress{Rono-Hills}
 \city{Doimukh} 
 \state{Arunachal Pradesh}
 \country{India}}
\author{Huifen Chan}
\affiliation{%
  \institution{Tsinghua University}
  \streetaddress{30 Shuangqing Rd}
  \city{Haidian Qu} 
  \state{Beijing Shi}
  \country{China}
}

\author{Charles Palmer}
\affiliation{%
  \institution{Palmer Research Laboratories}
  \streetaddress{8600 Datapoint Drive}
  \city{San Antonio}
  \state{Texas} 
  \postcode{78229}}
\email{cpalmer@prl.com}

\author{John Smith}
\affiliation{\institution{The Th{\o}rv{\"a}ld Group}}
\email{jsmith@affiliation.org}

\author{Julius P.~Kumquat}
\affiliation{\institution{The Kumquat Consortium}}
\email{jpkumquat@consortium.net}

% The default list of authors is too long for headers}
% \renewcommand{\shortauthors}{B. Trovato et al.}
\renewcommand{\shortauthors}{}


\begin{abstract}
Graphs in many real-world applications, such as social networks and business to business networks, are not deterministic but uncertain. Research in areas such as social science and viral marketing requires open access to these uncertain graphs, but publishing these datasets often risks exposing sensitive and private data. Current works mainly concentrate on privacy issues with deterministic graphs. The uncertain scenario is overlooked. 

In this paper, we study the problem of publishing uncertain graphs under the syntactic anonymity guarantee. We first demonstrate the limitations of conventional methods in the uncertain scenario. By disregarding the possible world semantics of uncertain graphs, they significantly disrupt the stochastic structure. We develop a syntactically private algorithm, Squid. Squid integrates the possible world semantics into the core of the anonymization. It enables a fine-grained, uncertainty-aware control over the injected noise in a novel way. We apply our method to real uncertain graphs and show its efficiency and practical utility.
\end{abstract}

%
% % The code below should be generated by the tool at
% % http://dl.acm.org/ccs.cfm
% % Please copy and paste the code instead of the example below. 
% %
% \begin{CCSXML}
% <ccs2012>
%  <concept>
%   <concept_id>10010520.10010553.10010562</concept_id>
%   <concept_desc>Computer systems organization~Embedded systems</concept_desc>
%   <concept_significance>500</concept_significance>
%  </concept>
%  <concept>
%   <concept_id>10010520.10010575.10010755</concept_id>
%   <concept_desc>Computer systems organization~Redundancy</concept_desc>
%   <concept_significance>300</concept_significance>
%  </concept>
%  <concept>
%   <concept_id>10010520.10010553.10010554</concept_id>
%   <concept_desc>Computer systems organization~Robotics</concept_desc>
%   <concept_significance>100</concept_significance>
%  </concept>
%  <concept>
%   <concept_id>10003033.10003083.10003095</concept_id>
%   <concept_desc>Networks~Network reliability</concept_desc>
%   <concept_significance>100</concept_significance>
%  </concept>
% </ccs2012>  
% \end{CCSXML}
% 
% \ccsdesc[500]{Computer systems organization~Embedded systems}
% \ccsdesc[300]{Computer systems organization~Redundancy}
% \ccsdesc{Computer systems organization~Robotics}
% \ccsdesc[100]{Networks~Network reliability}


% \keywords{ACM proceedings, \LaTeX, text tagging}


\maketitle

\section{Introduction}

\label{sec:Intro}
Graph privacy issues are becoming increasingly important for many applications such as business to business, social networks and instant-messaging networks. 
This is evidenced by the fact the responsible management of sensitive personal information is explicitly mandated through regulations such as European Union's General Data Protection Regulation.
% There are public concerns on how data companies collect personal information and make it available to marketers. 
Users are afraid that the sensitive personal information can be exploited in harmful ways (e.g., cyberstalking).
Meanwhile, data companies have strong fears of violating privacy regulations. 
Accordingly, there has 
been considerable interest in publishing graph data or analysis result under certain privacy guarantees~\cite{Liu_Towards_2008,Wang2011,Liu_Privacy_2009,Nguyen_Anonymizing_2015,Sala_Sharing_2011,Xiao_Differentially_2014,lee2011}. 
However, the majority of works has been limited to the deterministic scenario. 

In some applications, the graph may be probabilistic by nature. 
For example, in social networks, an edge between two individuals corresponds to an interaction or the influence that is predicted through machine learning models characterized by some level of uncertainty, which can thus be conveniently presented as the probability of the existence of that edge~\cite{Adar_Managing_2007,Kempe_Maximizing_2003}. In these situations, the problem of privacy protection becomes far challenging. 

\textbf{Uncertainty: critical privacy factors}~~First, edge uncertainty contains sensitive information about individuals such as a user's influence in a community or over another individual. 
For example, if we consider a small community, the members and the link structures of the members are known to any participant who is also a member of the community. However, the edge uncertainty such as the ``trustworthiness" of user A to user B should be private.
Second, the extra release of edge uncertainty makes the user identity more vulnerable. 
For example, if we consider a small community, several ``trustworthiness" connections between members are known to an intruder who is also a member of the community. The intruder can match the released probabilistic graph with known information, and serves in privacy attacks.  

\textbf{Uncertainty: critical utility factors} Meanwhile, these probabilistic graph are valuable for research and applications such as marketing and advertising, friend recommendation and modeling the structure and dynamic where uncertainty plays a indispensable role. For example, effective advertising needs to leverage the trust and influence relationships among users (uncertainty) as they may greatly impact users' behavior, as illustrated in Figure~\ref{fig:motivationExample}.  The primitive of these applications is to categorize and compute reachability between any two nodes. Therefore, it is these properties of edge uncertainty that must be preserved. 

\textbf{The limitation of existing methods}~~
In these situation, existing solutions for privacy protection graph publication and analysis do not apply because of the ignorance of edge uncertainty. 

Together, it calls for fundamental anonymization primitives for publishing uncertain graphs which, as ever-argued, would become different from the deterministic ones.

\textbf{Challenge} 
The challenges in privacy-preserving uncertain graph publishing are both semantics and computation driven. 

\textbf{What to xxxx}
% From the perspective of the semantics, there is no privacy notation and utility loss metrics over the probabilistic scenario.   First, it is much more challenging to model attacks where the background knowledge and released data both are probabilistic. 
% Second, it is much more challenging to model the utility loss over 


\textbf{How to anonymize uncertain graph efficiently ?}~~While many graph anonymization algorithms such as k-automorphism are intrinsically hard problems, even the simplest degree anonymization by  contractions with minimal loss of information is known to be NP-hard in the deterministic cases~\cite{}. In the probabilistic scenario, edge modifications are no longer limited to edge addition and deletions but can be infinite probability deviation,hence, more expensive over uncertain graphs. Therefore, exact computation is infeasible with today's large-scale graph data. 
Meanwhile, approximation methods designed for deterministic graphs does not guarantee efficiency and in uncertain graphs. For example, xx xx  

\textbf{Our contribution} In this paper we develop novel strategies for sanitizing uncertain graph with dynamic sampling, and filtering strategies. 
One key idea is that Squid captures the structure
of the original decentralized graph, by incrementally identifying
and refining clusters of connected nodes under local differential
privacy. 
To do so, Squid iteratively xxx.
After obtaining such node clusters, it applies a graph
generation model that utilizes such groups to generate a representative
synthetic social graph. Also, we describe techniques to
optimize critical parameters of xxx to improve the utility of the
generated synthetic social graph.


To validate the effectiveness of Squid, we present an extensive
set of experiments using three real uncertain graphs in various
domains, and two different use cases: (i) statistical analysis of the
 graph structure, (ii) influence propagation. 
 
The experiment results show that synthetic
 generated using Squid obtains high utility for all
use cases and datasets, whereas baseline solutions fail to capture
competitive utility in most settings except for the few that they are
specifically optimized for. 

Our main contributions are summarized as follows: 
\begin{itemize}
\item We formulate the problem of synthetic data generation of uncertain graph under syntactic anonymity while keeping the data utility in mind. To this end, we propose a reliability-driven utility loss metric, which evaluates the connectivity difference in the context of the entire graph and also utilizes the possible world model.  

\item We propose Squid, a novel and effective randomized algorithm to synthetic uncertain graph generation and describe methods for scaling computation. Boosted by the hybrid of uncertainty-aware heuristics, it excels in identifying a population of synthetic results with good quality efficiently. 

\item We conduct a comprehensive experimental study using several
real datasets and use cases. The results demonstrate that
Squid improves the utility of
obtained result, while providing users with their desired privacy guarantees. 
\end{itemize}

\begin{figure}
    \centering
    \includegraphics[height=2.3cm] {figure/motivationExampleWideWithApplication.pdf}
    \caption{ An Motivation Example}
    \vspace{-10pt}
    \label{fig:motivationExample}
\end{figure}

% \begin{figure}[t]
%     \subfigure[Social Trust Network]{\label{fig:socialNetwork}
%       \begin{minipage}[l]{0.40\columnwidth}
%         \centering
%         \includegraphics[height=2.3cm]{ill/SocialNetwork.pdf}
%       \end{minipage}
%       }
%     \subfigure[B2B Network]{\label{fig:b2bNetwork}
%       \begin{minipage}[l]{0.40\columnwidth}
%         \centering
%         \includegraphics[height=2.3cm]{ill/B2BNetwork.pdf}
%       \end{minipage}
%       }
%     \vspace{-6pt}
%     \caption{Real uncertain graphs with privacy concerns.}
%     \vspace{-10pt}
%     \label{fig:motivation}
% \end{figure} 


The rest of the paper is organized as follows. In Section~\ref{sec:relatedWork}, we summarize related works, point out the limitation of existing methods, and clarify our distinct privacy goal. In Section~\ref{sec:notation} we formulate the uncertain graph-anonymization problem. Sections~\ref{sec:soa}--\ref{sec:method} present our anonymization approach for privacy-preserving uncertain graph sharing.  In Section~\ref{sec:ex} we apply our method to several real-world uncertain graphs and demonstrate its performance, practical utility, and efficiency. 
\section{Related Work}
\label{sec:relatedWork}
A significant amount of prior work has been done on protecting the privacy of network datasets.
The comprehensive survey is out of the scope of this paper. 
Here, we briefly summarize related work and clarify our privacy goal. 

\textbf{Syntactic Privacy.}~~
Early works on privacy-preserving network releasing focus on developing anonymization techniques.
Many of them modify the graph structure in subtle ways that guarantee privacy but keep much of graph structure for release. 
The released graph is available for all the analysis tasks. 
These approaches usually provide privacy protection against specific de-anonymization attacks. 
Most of them employ syntactic privacy models derived from $k$-anonymity~\cite{Sweeney:2002:KAM:774544.774552} which requires creating $k$ same entities ({\eg} neighborhoods, degree nodes) to blend victims. 

Related anonymization methods can be classified into four main categories: (1) Clustering-based generalization~\cite{Hay_Anonymizing_2007,Bhagat_Class_2009,hay2010resisting}; (2)~{\em Edge modification}~\cite{Liu_Towards_2008, Zhou_Preserving_2008, Zou:2009, Wang2011, Wu_k_2010, Skarkala_Privacy_2012}; 
(3)~{\em Edge randomization}~\cite{Liu_Privacy_2009,Ying_Randomizing_2008, Ninggal_Utility_2015};
and~(4)~{\em Uncertainty semantic-based modifications} which add uncertainty to some edges and thus converting the deterministic graph to an uncertain version for anonymity~\cite{Boldi_Injecting_2012, Nguyen_Anonymizing_2015}. 

In the first category, Hay {\etal}~\cite{Hay_Anonymizing_2007} proposed to generalize a network by clustering nodes and only publish the hyper-graph ($\#$ of nodes in each partition with $\#$ of edges within and across partitions). Campan {\etal}~\cite{Campan2008} studied the attributed graph case with a similar solution. 
Cohen {\etal}~\cite{Cohen2013} presented a sequential clustering algorithm with better utility preserving. While, Cormode{\etal} \cite{Bhagat_Class_2009} payed attention to attributed-based matching attack. To this end, their method marks the mapping by clustering the nodes and corresponding real-world entities into groups. 

In the second category, Liu {\etal}~\cite{Liu_Towards_2008} focused on resisting degree-based entity re-identification attacks. They propose to add and delete edges to pursue $k$-degree anonymity. Zhou {\etal}~\cite{Zhou_Preserving_2008} consider stronger re-identification attack based on radius-one sub-graph. Zhou~{\etal}~\cite{Zou:2009} assume that the adversary knows the compute graph. Their algorithms use edge addition and deletion to make graph $k$-Automorphism.  

In the third category, Hay {\etal} \cite{Liu_Privacy_2009} study the use of random perturbation for identity obfuscation. They consider the basic degree-based re-identification of nodes. Besides, they propose to quantify the level of anonymity that is provided for the given node $v$ in the real network by the perturbed graph as the inverse of the maximum of the belief probability $Pr(v|u)$. Ying {\etal} \cite{Ying_Randomizing_2008} compare random perturbation methods to the method of $k-$degree anonymity. Their experiments show the deterministic edge modification methods for $k$-degree anonymity preserves the graph structure better than random perturbation methods.

The uncertainty semantic-based approaches are known as the state-of-art ones because of their excellent privacy-utility trade-off, brought by the fine-grained perturbation leveraging the uncertain semantics. 
% add two-sentences about the works 
Our method belongs to the fourth category while in the wider context. 

 
\textbf{Differential Privacy.}~~
The dependence of adversary knowledge makes graph anonymization methods are vulnerable to attackers with strong background knowledge than assumed. Such fact has simulated the use of differential privacy for more rigorous privacy guarantees. 
The recent research on applying differential privacy to graph data roughly falls into two directions. The first direction aims to release specific differentially private mining results, such as degree distributions, sub-graph counts, and frequent graph patterns~\cite{Xiao_Differentially_2014, Day:2016}. These methods only publish query result. However, there are many situations in which answering statistical queries simply does not achieve the purpose of sharing the graph.  
The second direction aims to share the meaningful graph. Most research in this direction~\cite{Sala_Sharing_2011,Proserpio_2012} projects an input graph to dK-series and ensures differential privacy on dK-series statistics. Later, private statistics are then either fed into generators or MCMC process to generate a fit synthetic graphs. However, current techniques are still inadequate to provide desirable data utility for many graph mining tasks. Wang {\etal}~\cite{Wang_2013} propose to project a graph to the spectral domain and inject noise into the eigenvalues and eigen vector. This approach achieved significant improvement in efficiency, which, is still not able to obtain useful data utility. Xiao {\etal}~\cite{Xiao_Differentially_2014} present a solution based on structural inference over the hierarchical random graph model. This approach achieved the reasonable utility over real-life graph datasets. 

As ever discussed, the state-of-art of graph privacy research has been limited to publishing data or analysis result of deterministic graphs. 
The uncertain scenario is unexplored. 
\subsection{Our Privacy Goal}
In this paper, we try to move this line of research one step forward, from the deterministic context to a broader probabilistic context.
This study focuses on generating a synthetic graph data from a real, probabilistic one under syntactic privacy guarantees. Such a synthetic graph enables scientist to draw meaningful insight while protecting the participants involved and the data collectors against risks of privacy violation. 

Another critical component is the choice of privacy notation. 
The concept of different privacy (DP) was proposed as a strong privacy measure such that sensitive information of individuals is kept private in data analysis and publishing process~\cite{Sala_Sharing_2011,Xiao_Differentially_2014}.
Immune to various privacy attacks, DP and its off-springs offers a guarantee bound $\epsilon$ on the loss of privacy due to the data release. 
While the data collector, such as e U.S. Census Bureau, has encountered many challenges in deploying differential privacy. These challenges includes, but not limited to, the difficulty in setting the value of the privacy-loss parameter, $\epsilon$ (epsilon), the lack of release mechanisms that align with the needs of data users, and the expectation on the part of data users that they will have access to micro-data~\cite{Simson:2018}. 
Recall that the trade-off between between statistical
accuracy and privacy loss is at the heart of privacy protection. 
However, Differential privacy lacks a well-developed
theory for measuring the relative impact of added noise on the utility of different data applications, tuning equity trade-offs, and presenting the impact of such decisions. 

In contrast, the notion of syntactic privacy can be defined and understood based on the data schema. And, its parameters have a clear privacy meaning that can be understood independent of the actual data. Moreover, they have a clear relationship to the privacy regulation of individual identifiability (e.g., General Data Protection Regulation).
Meanwhile, the specific utility measure can be incorporated into the privacy mechanism so that the usefulness is higher for the same privacy loss budget, allowing the overall privacy-loss budget to be better deployed.
Hence, we focus on sharing uncertain graphs with syntactic anonymity. 
\input{notation.tex}
\section{The State-of-Art Approach}
\label{sec:soa}
Before presenting our solution {\methodName}, we first describe the state-of-art deterministic graph anonymization approach ({\soaName})~\cite{Boldi_Injecting_2012}.
The purpose of describing it is to separate the basic framework with the contribution of {\methodName}. 
They differ in the search strategy of sanitized candidates.  

\subsection{Overview}~~
The {\soaName} method obfuscates the (deterministic) graph data by adding or removing edges \emph{partially}. 
For each edge $e$, it assigns a probabilistic deviation $r_{e} \in [0,1]$, where $r_{e} \leftarrow R(\sigma)$. 
In particular, the uncertainty injecting scheme proceeds as follows:
\begin{equation}
    p(e) =
    \begin{cases}
         1-r_{e}  & e \in E \\
         r_{e}    & otherwise 
    \end{cases}
    \label{eq:inject}
\end{equation}
Generally, it transfers edge existence from existing edges to non-existing ones for identification obfuscation.   

\begin{figure}[htb]
  \centering
        \includegraphics[width=0.8\linewidth]{figure/std_2.jpg}
  \vspace{-5pt}
  \caption{Illustration of the obfuscation effect brought by larger values of standard deviation $\sigma$.}
  \vspace{-5pt}
  \label{fig:std}
\end{figure} 

For the high utility of the obfuscated graph, smaller values of the parameter $r_{e}$ should be favored.  
The most widely known member of generating distribution $R({\sigma})$ is the truncated normal distribution with mean 0 and variance $\sigma^2$. 
In principle, $R$ could be any distribution.
As the standard deviation $\sigma$ decreases, a greater mass of $R_{\sigma}$ will concentrate near $r_{e}=0$.  
As illustrated in Figure~\ref{fig:std}, smaller values of $\sigma$ generally contributes towards better utility preserving, while at the same time they provide lower level of obfuscation. 
Larger values of $\sigma$ have the opposite effect.
Then, the amount of injected noise and consequent structural distortion will be smaller. 
Aiming at the high utility, {\soaName} aims at injecting the minimal amount of uncertainty need to achieve the necessary obfuscation. 
As outlined in Algo~\ref{alg:obf}, it computes the minimal amount of uncertainty via a binary search on the $\sigma$ value. 
\input{mainRoutine.tex}

The function genObf, which aims at generating {\keobf} instances using a given standard deviation parameter $\sigma$, determines the search flow. It either returns the generated sanitized instance or the failure signal.
The search starts with an initial guess of an upper bound $\sigma_{u}$, which is iteratively doubled until a {\keobf} instance is found. Then, the binary search is performed over the range $[0,\sigma_{u}]$. The binary search terminates until the search interval is sufficiently short. The algorithm outputs the best found sanitized output  (the last one that was successfully generated; the one with the smallest $\sigma$).

\subsection{GenObf Function}
% It is difficult to find {\keobf} sanitized solutions using a given parameter $\sigma$ over the intractable search space. 
The genObf function separates the search process into running two independent modules: (1) uncertain noise generative models and (2) privacy tests. 
The first constructs a utility-preserving noise generative model. 
Meanwhile, the privacy test aims to safeguard the privacy of generated obfuscation. 
Overall, it leverages random search to alleviate the combinational intractability.  
Particularly, multiple randomized attempts are performed. 
Every obfuscated output is subject to this privacy test. 
Iff all the attempts fail, {\genobf} returns failure sign. Otherwise, it returns the found {\keobf} instance.

Each construction attempt begins with selecting a subset of edges subject to alteration. 
Then, it assigns the deviation among selected edges and injects uncertainty. 
While, the randomization process heavily relies on the heuristic.  
In particular, {\soaName} suggests calibrating the perturbation applied to an edge $e$ according to the ``uniqueness" of the two nodes $u$ and $v$. 
In brief, if both $u$ and $v$ are common nodes {\wrt} the property, then $r_{e}$ should be very small; 
on the other hand, if $u$ and $v$ are outliers, then $r_{e}$ should be higher. 
Meanwhile, edges need to be sampled with the higher probability if they are adjacent to outliers. 

\subsection{Limitations} 
The {\soaName} method provides the desired level of privacy guarantee with the small change in the graph, thus maintaining high utility.
However, this method has two critical weaknesses in the probabilistic context:
(1) The design heavily tailored towards the deterministic scenario {\eg} it assumes the existence of edges is binary (0,1). Thus, it fails to handle uncertain graphs where the existence of edges is probabilistic. 
All the operators, including edge selection and alteration, need to be integrated with possible world semantics carefully.
(2) Its scheme does not consider the structural relevance of edges in critical edge selection/alteration steps, which leads to unnecessary structural distortion.
We are left asking the following questions, \emph{how to generalize existing methods to the probabilistic context?} and \emph{how to get a better trade-off between privacy and utility in the probabilistic context?} 
\input{squid.tex}
\input{squid_ex/exp.tex}
\section{Conclusion}
In this work, we first examine the overlooked problem of privacy-preserving uncertain graph sharing. 
We provide a generalized scheme, {\methodName}, which seamlessly integrates edge uncertainty into core anonymization steps.
The scheme uses a randomized algorithm boosted by the hybrid of utility-aware heuristics. 
It excels in identifying sanitized uncertain graphs with excellent quality. 
Experiments on three real-world datasets verify its effectiveness and practical utility.
There are many potentials to explore further.  
The independent model discussed in this work is one of the simplest models with uncertainty, it naturally ignores the correlation among various graph components. 
We leave the conditional probability model as a future extension. 
Another extension is to investigate sharing uncertain graphs in the differentially private manner.

\bibliographystyle{ACM-Reference-Format}
\bibliography{refs} 

\end{document}
