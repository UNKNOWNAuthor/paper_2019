\subsection{Uncertain Graph Statistic Preserving}
In the set of experiment, we focus on evaluating their performance concerning graph statistic preserving. 
The evaluation includes two groups of graph statistics. 
The first group includes node separation statistics that quantify the interconnectivity and density of the overall graph.   
This group includes metrics such as Shortest Path Length, Reliability and Graph Diameter. 
They are very expensive because they involve all-pairs shortest path computations. 
To overcome this problem, we use Hyper ANF~\cite{Boldi_Rosa_Vigna_2011} to approximate shortest path-based statistics.
The second group includes degree-based statistics such as Average  Node Degree and Degree Distribution.
These topological statistics that characterize how degrees are distributed among nodes. 
Our results are highly consistent across our pool of graph statistics.
For brevity, we only report  Reliability, Shortest Path Length and Node Degree as their representatives. 

\textbf{Node Separation Statistic.} 
Figure~\ref{fig:rd} shows the incurred reliability discrepancy. 
We can see that larger $k$ introduced more significant connectivity distortion due to the fact that more noise was added to achieve the desired level of anonymity.   
These observations are consistent across biological (PPI) and social  networks (BK and DBLP).

{\methodName} method preserves very well reliability in all datasets, followed by the CA approach with the similar objective function. RA performs poorly for DBLP. 
For example, in all the dataset ($k=300$), the reliability discrepancy introduced by {\methodName} is well below 0.1 whereas the ones added by CA is below 0.2. The reliability discrepancy introduced by RA on PPI, BK, DBLP is around 0.2, 0.2, 0.4 respectively. We also observe that as the size of graph increases (PPI $\rightarrow$ DBLP), the performance gap becomes larger and larger.

Note that CA and RA schemes deteriorate data utility due to the disregarding the possible world semantics. 
For example, in the case, $k=100$ (weak privacy guarantee required little noise), CA and RA incur relatively large connectivity distortion.  
The representative extraction step of RA introduces noise and results in cumulative errors in the anonymization step. Consequently, sanitized results differ from the original ones. 
The CA scheme fails to reflect the connectivity of uncertain graph correctly. Thus, it produces inferior results even with the similar objective function.  

Figure~\ref{fig:pathd} shows the relative error of \emph{Shortest Path Distance} of sanitized output graphs. In general, the ranking of anonymization algorithms regarding preserving path-based statistic is analogous to that for reliability (see Figure~\ref{fig:rd}) because the shortest path distance between a node pair $\langle u,v\rangle$ is highly correlated to two-terminal reliability. 



\textbf{Degree-based Statistic.}~Figure~\ref{fig:dd} shows the error of Node Degree values on PPI, BK and DBLP compared to their sanitized outputs. 
The obfuscated output of {\methodName}, CA capture well Node Degree in all datasets; 
our method {\methodName} is consistently better.    
In the largest dataset DBLP, the degree error $0.2$ is much lower than $1.08$ imposed by the two-phase anonymization scheme (RA) in the same level of obfuscation ($k=300$).  
As with previous experiments, the performance gap increases as the graph size increases.

\textbf{Summary.} 
Our experimental results show that sanitized outputs generated by {\methodName} exhibit structural features close to those of their original uncertain graphs. 
Results show that we can effectively balance the utility and privacy in the probabilistic graph context.
The result is encouraging because we can eliminate the noise by moving from the reliability model to a more accurate graph model incorporating with the possible world semantics.